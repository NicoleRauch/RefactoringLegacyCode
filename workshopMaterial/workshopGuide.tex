\documentclass[a4paper,fleqn,titlepage,11pt]{article}
\usepackage{t1enc} 
%%\usepackage{german}

%%\selectlanguage{german}


\pagestyle{myheadings}
\pagenumbering{arabic}
\setlength{\parindent}{0mm}		% Einzug f�r die erste Zeile im Absatz

%% Seitenformat:
\setlength{\oddsidemargin}{0cm}		% li. Randabstand auf re. Seiten
\setlength{\textwidth}{16cm}		% Breite des Textes
\setlength{\topmargin}{-0.75cm}		% Abst. Oberkante Blatt - Oberk. Header
\setlength{\headheight}{30pt}		% H�he des Headers
\setlength{\headsep}{5mm}		% Abst. Header - Text
\setlength{\topskip}{5mm}		% Oberkante Text - Grundlinie 1. Z.
\setlength{\textheight}{23.5cm}		% H�he des Textes
%%\setlength{\footheight}{0cm}		% H�he des Footers
\setlength{\footskip}{1cm}		% Abst. Unterk. Text - Unterk. Footer



\begin{document}

\thispagestyle{empty}

\begin{center}
\LARGE
Workshop Guide

\vspace{1cm}

Restructuring Code: From ``Push''  To ``Pull''
\end{center}

\normalsize

\vspace{0.6cm}

%%%%%%%%%%%%%%%%%%%%%%%%%%%%%%%%%%%%%%%%%%%%%%%%%%


\section{The Initial Situation}

This describes the general situation in which the restructuring pattern is applicable.

\subsection{Aim of the Code}

The code produces a sequence of data objects, each of which may hold several data (e.g.~several numbers).

\subsection{Initial Shape of the Code}

\begin{itemize}
\item one class that holds one input data (here: the class \texttt{Transaction})
\item a sequence of these input data elements (here: \texttt{List<Transaction>})
\item very few classes with logic (here: the class \texttt{PushingBalancesCalculator})
\item one class that holds one of the output data entries in the sequence (here: \texttt{BalancesOfMonth})
\item a sequence of these output data elements (here: \texttt{List<BalancesOfMonth>}) which is constructed early in the program run
\item in the logic class(es), there are two interlaced loops: 
\begin{itemize}
\item initially, the sequence of output data is not filled
\item the outer loop iterates over the sequence of data objects (here: the months for which the output data is to be determined)
\item the inner loop calculates the data for each data object (here: the sum of the transactions for each month)
\item at the end of the inner loop, the calculated data is written to the data object (``push'')
\item finally, the sequence of output data is filled
\end{itemize}
\item the logic class(es) may contain multiple of these interlaced loops in sequence, where a subsequent loop may access and modify the output data that a previous loop has written to the sequence of output data
\end{itemize}

\subsection{Inspection of the Code}
The code doesn't look too bad. It has some issues. The algorithm is not easy to understand. Testing is only possible on an integration base. There is no isolation or abstraction levels.

\section{The Overall Goal}

Let's make the code crisper. That means: easy to read, understand, test and extend.

The overall goal is to extract the calculations of the individual values into separate methods.

This can be broken up into three major parts:

\begin{enumerate}
\item Disentangle the for loops, isolate the loop body and extract it
\item Build one data object for each loop iteration (instead of reusing the same object over and over again)

\item Move the extracted loop body to its own class, isolate the calculations of the distinct values into separate methods
\end{enumerate}

In the following, we will split up these two parts into fine-grained steps that can be applied to the code.

We do this by:
\begin{itemize}
	\item Defining one interaction point with outside objects (Step 1)
	\item Separating value calculation from iteration (Steps 2-3)
	\item Encapsulating the calculation object (Steps 4-5)
	\item Separating calculation for each value (Step 6)
	\item Eliminating unwanted state (Step 7)
	\item Tidying up (Step 8)
\end{itemize}

Restrictions / Rules:

Change one thing at a time.

\begin{itemize}
\item Right now, we neither want to change the outside world nor our calculator class API, only the internals of our calculator class. The result objects (\texttt{BalancesOfMonth}) represent the outside world in our example. Therefore, we will leave the result objects and the \texttt{fillData} method signature untouched. 
\end{itemize}


\textbf{Note:} The workspace with the step number contains the \textbf{solution} to the step!

\part{Disentangle the For Loops and Extract the Loop Body}

Overall Goal:

Extract most of the outer for loop body into a separate method.

\section*{Pre-Arrangements (Workspace: Push)}

Tidy up the code, rename local variables to make their names more descriptive.
Push the loops as far to the outside of the method as possible (move as much code as possible inside the loop bodies).

\section*{Step 00: Decoupling from the outside world (From Local Variables To A Data Object)}

Observations:
\begin{itemize}
\item The outer loop iterates over the list of result objects that need to be filled with the calculated data by the method.
\item The method calculates two values for each result object: balance and averageBalance. 
\item These two values cannot both be returned from a single method.
\end{itemize}

Solution:

Create an object that will transport the calculated values. 

In this step, the intermediate object will only hold the values, it will not contain any logic. The intermediate object will have one setter that sets all calculated values simultaneously. It will also have getters, one for each of the calculated values.

In the logic class, we will create an intermediate data object at the beginning of the calculation. Instead of directly writing the calculated values to the output data structure, we write them to the intermediate object and then read it again from the intermediate object and write it to the output data structure. We reuse the same intermediate object for each run through the loop.

\section*{Step 00a: Prepare for method extraction Part I: Resolving ambiguous return values}

Observations:
\begin{itemize}
\item If we tried to extract the body of the outer loop into a method, we would be told that this method manipulates two variables which cannot both be returned, namely balance and latestBalance.
\end{itemize}

Solution:

Move the two local variables into the loop body.

Regarding balance:

If we write balance to balanceAndAverage at the end of the loop, it is no problem if we read balance from balanceAndAverage at the beginning of the loop.

Regarding lastBalance:

At the end of the loop, lastBalance equals balance, so it is no problem if we set lastBalance to balance at the beginning of the loop.

\section*{Step 01: Prepare for method extraction Part II: Purifying the dependencies}

Observations:
\begin{itemize}
\item If we extracted the body of the outer loop into a method, this method would depend on our outer world, i.e. \texttt{BalancesOfMonth}.
\item We only need the date from the \texttt{BalancesOfMonth} object, namely for calculating the current ultimo.
\item If we extracted the body of the outer loop into a method, this method would depend on state of our object, namely \texttt{transactions} (inside \texttt{transactionsOfMonth}).
\end{itemize}

Both is undesirable.

Solution:

Move the transaction filtering to the top of the loop.

Extract the date from the \texttt{BalancesOfMonth} object, move it to the top of the loop and use it in the rest of the method body.


\section*{Step 02: Isolation of calculation from iteration (Extract the Value Calculation into a Separate Method)}

Observations:
\begin{itemize}
\item We can now extract the major part of the method body into a separate method.
\end{itemize}

Solution:

Let's do it!

\part{One object for each loop iteration}

Overall goal:

Build one data object for each loop iteration (instead of reusing the same object over and over again).


\section*{Step 03: Separation of concerns (Move the Calculation Method to the Intermediate Class)}

Observations:
\begin{itemize}
\item The extracted method uses the intermediate data object as in-out-parameter, so it can read values from the previous iteration (if required) and write the calculation results into the data object.
\item The extracted method only operates on this parameter but not on the state of the class it is in. {\em (Code smell: Feature Envy)}
\end{itemize}

Solution:

The extracted calculation method can be moved to the class of the intermediate object. (If this is not done via an automated refactoring. This changes the invocation site of the method as well.)

And don't forget to manually move the method calculateProportionalBalance as well! - After that you can (automatically) change the signature of the extracted method to remove the PushingBalancesCalculator parameter.

Modified objects:

PushingBalancesCalculator

BalanceAndAverage


\section*{Step 04: Use New Instance of Intermediate Object for Each Execution of the Inner Loop Body}

Observations:
\begin{itemize}
\item The for loop reuses the same BalanceAndAverage object in each loop iteration.
\item The only value that needs to be transported from one iteration to the next is the balance of the preceding iteration, which is already retrieved anyway.
\end{itemize}

Solution:

We explicitly pass the preceding balance to the calculation. We have to fetch it from the previous result before creating a new instance.

In this step, we want each execution of the loop body to create its own intermediate object. If the value calculation uses values from a previous calculation, we need to take care of this and pass on the required values from the previous intermediate object to the calculation method, possibly widening its signature.

Modified methods:

PushingBalancesCalculator.fillData()

BalanceAndAverage.calculateValues()

\part{From Calculating the values up-front to calculating the values on-demand}

Overall Goal:

Currently, all values are calculated up-front with an explicit command from the outside, no matter whether they are needed or not. It would be much nicer if our object could calculate the values on-demand when they are needed.


\section*{Step 5: Remove the Parameters from the Value Calculation Method}

Observations:
\begin{itemize}
\item We still need to explicitly invoke the value calculation on the intermediate object. 
\item We pass all required inputs into the calculation method.
\end{itemize}

Solution:

In order to get rid of the explicit calculation method, the first step would be to pass the input values into the constructor of the result object, thus turning these inputs into state for the calculating object.

\section*{Step 6: Split up the Value Calculation}

Observations:
\begin{itemize}
\item So far, we still calculate all values at once, in the same method. 
\item This method invokes a setter that writes all values into the intermediate object's attributes in one go (we introduced this setter in Step 1).
\end{itemize}

In this step, we want to split up the value calculation and value setting. In order to do so, we duplicate the calculation method such that we get one method for each of the values that we want to calculate. We name these methods accordingly, e.g.~\texttt{calculateValue1}, \texttt{calculateValue2} and so on.

In the calculation method we replace the invocation of the original calculation method by the invocations of the individual value calculation methods.

Now we can remove all code from each of the methods that is not related to calculating the value for which the method is responsible. Each of these methods only sets the value it is responsible for, and it directly sets the value into the corresponding field.

When we are done, we can remove the setter that sets all values simultaneously.

Modified:

BalanceAndAverage

\section*{Step 6a: Introduce Return Values into Calculation Methods}
Observations:
\begin{itemize}
\item The two calculation methods do calculation and setting fields. 
\end{itemize}

We let the calculation methods return values and set the fields inside the calling method.

Modified:

BalanceAndAverage

\section*{Step 7: Turn the Value Calculation Methods into Getters}

Observations:
\begin{itemize}
\item In the value calculation methods, we do not write to fields any more. Instead, we just return the calculated values.
\item These methods are still invoked from the outside (via the invocation method that invokes all of the single calculation methods).
\item We have getters that read the values from the fields and return them.
\end{itemize}

Solution:

In the getters of the intermediate object, we do not return the field values but instead the results of the value calculation methods.

Now we can remove the fields (they are no longer needed) and the value calculation method invocations in the overall calculation method (they no longer have any side-effects, so it makes no difference whether we invoke them or not).

Finally, we can inline the value calculation methods into the getter methods. This way, we have getters for the individual values of the intermediate object which describe exactly how each value is being calculated.

Modified:

BalanceAndAverage

PushingBalancesCalculator (calculation method invocation is removed).

\section*{Step 8: Simplification}

Observations:
\begin{itemize}
\item The calculation of the average balance is quite complicated.
\item The calculation of the average balance is isolated and easily testable.
\end{itemize}

Now that we can see each value calculation in isolation, we often notice that we can perform simplifications of the algorithms. 

Solution:

We can simplify the algorithm or even rewrite it (with TDD if we like). It is now easy to clarify the business aspects of this logic and to develop a new algorithm. The average is transformed from vertical slicing to horizontal slicing. The old algorithm adds each transaction based on how many days it lasts into the future. This is the reason why we need to memorize the old date, the new date, the old amount. Also this requires special handling for the last transaction in the month. The new algorithm calculates for each transaction the resulting transaction value on the base of remaining days until ultimo and adds the transactions's amount to the aggregate.

\section*{Final Step: Achieving the ``Pull'' Structure}

Observations:

\begin{itemize}
\item The outer for loop from the beginning still uses the same variable for the calculation objects.
\item After one loop iteration, the calculation object is gone. We cannot access the results later on.
\item If we want to access the results again at a later time, we need to recalculate everything again and again.
\end{itemize}

Solution:

Transform the calculation objects into a chained structure of calculation object (called Month) where each month can access its predecessor's values.

Create this chained structure up-front. Fill it with all relevant input data.

In the for loop, simply access the calculated results of the months.

Introduce caching of the calculated results in order to avoid performance issues.

--------

Pull
====

- Umwandeln der dummen Monatsdaten-Records in Monat-Objekte
- BestandUndDurchschnitt geht ebenfalls in dem Monat-Objekt auf
- Umwandeln der Liste von Monatsdaten in ein Monate-Objekt
- Verkettung der Monate statt Übertragen des Vormonatsbestands
- Bei Bedarf Caching der berechneten Werte in einem Wrapper-Objekt

\begin{itemize}
\item x
\end{itemize}

\section{The Final Situation}

Zielzustand:

- Berechnungslogik und Ergebnisstruktur sind ein und dasselbe geworden
- Intelligente Objekte statt stupider Datenhaltung

\begin{itemize}
\item x
\end{itemize}



\end{document}
